%%%%%%%%%%%%%%%%%%%%%%%%%%%%%%%%%%%%%%%%%%%%%%%%%%%%%%%%%%%%%%%%%%%%%%%%%%%
% Archivo para hacer derivaciones categoriales
%%%%%%%%%%%%%%%%%%%%%%%%%%%%%%%%%%%%%%%%%%%%%%%%%%%%%%%%%%%%%%%%%%%%%%%%%%%

% Configurar la compilación rápida como PdfLaTex + ver pdf

%%%%%%%%%%%%%%%%%%%%%%%%%%%%%%%%%%%%%%%%%%%%%%%%%%%%%%%%%%%%%%%%%%%%%%%%%%%
% Empieza el preámbulo
%%%%%%%%%%%%%%%%%%%%%%%%%%%%%%%%%%%%%%%%%%%%%%%%%%%%%%%%%%%%%%%%%%%%%%%%%%%

\documentclass[a4paper, 12pt]{article} %Tipo de documento y tamaño de letra
\usepackage[utf8]{inputenc} % Paquete de tipo de letra
\usepackage[T1]{fontenc} % Paquete de codificación
\usepackage[spanish]{babel} % Paquete de idioma
\usepackage[normalem]{ulem} % Paquete de codificación
\usepackage{lmodern} % La fuente

\usepackage{graphics} % Paquete para escalar las derivaciones
\usepackage{semantic} % Paquete para las derivaciones 
% La sintaxis para las derivaciones es la siguiente
% Si A es la premisa o conjunto de premisas, B la conclusión y C la regla de inferencia
% El código para la inferencia es \inference{A}{B}[C]
%%%%%%%%%%%%%%%%%%%%%%%%%%%%%%%%%%%%%%%%%%%%%%%%%%%%%%%%%%%%%%%%%%%%%%
% Empieza el documento

\begin{document}



\inference{\inference{Ringo}{n}[lex] \inference{duerme}{s{\backslash}n}[lex]}{s}[<apl]

\inference{\inference{\inference{Ringo}{n}[lex]}{s/(s{\backslash}n)}[Asc]\inference{duerme}{s{\backslash}n}[lex]}{s}[>Apl]


\scalebox{0.7}{ \inference{\inference{\inference{Macarena}{PN}[lex]\inference{\inference{\inference{riega}{(S{\backslash}PN)/PN}[lex]\inference{\inference{esa}{PN/CN}[lex]\inference{pequeña}{CN/CN}[lex]}{PN/CN}[comp]}{(S{\backslash}PN)/CN}[comp]}{(S/CN){\backslash}PN}[asoc]}{S/CN}[<apl]\inference{planta}{CN}[lex]}{S}[>apl]
}

\scalebox{0.6}{ \inference{\inference{\inference{Juan}{NP}[lex]}{S/(S{\backslash}NP)}[asc]\inference{\inference{\inference{rompió}{(S{\backslash}NP)/NP}[lex] \inference{el vidrio}{NP}[]}{S{\backslash}NP}[>apl] \inference{\inference{con}{((S{\backslash}NP){\backslash}(S{\backslash}NP))/NP}[lex] \inference{el martillo}{NP}[]}{(S{\backslash}NP){\backslash}(S{\backslash}NP)}[>apl]}{S{\backslash}NP}[<apl]}{S}[>apl]
}

\scalebox{0.7}{
\inference{\inference{Ringo}{N}[lex] \inference{\inference{comió}{(S{\backslash}N)/N}[lex] \inference{\inference{el}{N/N}[lex] \inference{\inference{sabroso}{N/N}[lex] \inference{hueso}{N}[lex]}{N}[>apl]}{N}[>apl]}{S{\backslash}N}[>apl]}{S}[<apl]
}



\scalebox{0.7}{\inference{\inference{\inference{Ringo}{N}[lex] \inference{\inference{\inference{comió}{(S{\backslash}N)/N}[lex] \inference{\inference{el}{N/N}[lex] \inference{sabroso}{N/N}[lex]}{N/N}[comp]}{(S{\backslash}N)/N}[comp]}{(S/N){\backslash}N}[asoc]}{S/N}[<apl] \inference{hueso}{N}[lex]}{S}[>apl]
}

\scalebox{0.7}{\inference{Ringo}{N}[lex] \inference{comió}{(S{\backslash}N)/N}[lex] \inference{el}{N/N}[lex] \inference{sabroso}{N/N}[lex] \inference{hueso}{N}[lex]}


\scalebox{0.7}{\inference{El}{N/N}[lex] \inference{tour}{N}[lex] \inference{mágico}{N{\backslash}N}[lex] \inference{misterioso}{N{\backslash}N}[lex] \inference{comenzó}{S{\backslash}N}[lex]}



\end{document}